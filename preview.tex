% !TEX TS-program = pdfLaTeX+MakeIndex+BibTeX
% !TEX encoding = UTF-8 Unicode

\PassOptionsToPackage{unicode}{hyperref}
\PassOptionsToPackage{naturalnames}{hyperref}

\documentclass[tg]{mdtufsm}

\usepackage[T1]{fontenc}
\usepackage{fix-cm}
\usepackage{url}
\usepackage{times, color}
\usepackage[utf8]{inputenc}
\usepackage{graphicx}
\usepackage{amsmath,latexsym,amssymb}
%\usepackage[hidelinks]{hyperref}
\usepackage[hidelinks,
            bookmarksopen=true,linktoc=none,colorlinks=true,
            linkcolor=black,citecolor=black,filecolor=magenta,urlcolor=blue,
            pdftitle={Automatização do Gerenciamento e Implantação de Instâncias AWS EC2 para Ensino de Computação Distribuída},
            pdfauthor={Cezar Augusto Contini Bernardi},
            pdfsubject={Trabalho de Graduação},
            pdfkeywords={Cloud computing, orquestração, computação distribuída, ensino, UFSM}
            ]{hyperref}
            
%\usepackage[brazilian]{babel}

%\usepackage{fontspec}
%\setmainfont{Linux Libertine G}

%%% PAGE DIMENSIONS
\usepackage[inner=30mm,outer=20mm,top=30mm,bottom=20mm]{geometry} 
\usepackage{epstopdf}
\usepackage{graphicx}
\usepackage{pdfpages}
% \geometry{margin=2in} % for example, change the margins to 2 inches all round
% \geometry{landscape} % set up the page for landscape

% \usepackage[parfill]{parskip} % Activate to begin paragraphs with an empty line rather than an indent

%%% PACKAGES
%\usepackage{amsfonts}
\usepackage{color}
%\usepackage{booktabs} % for much better looking tables
%\usepackage{array} % for better arrays (eg matrices) in maths
%\usepackage{paralist} % very flexible & customisable lists (eg. enumerate/itemize, etc.)
\usepackage{verbatim} % adds environment for commenting out blocks of text & for better verbatim
\usepackage{listings}
\usepackage{parcolumns}
\usepackage{siunitx}
%\usepackage{verbatim} % adds environment for commenting out blocks of text & for better verbatim
\usepackage{subcaption}
\captionsetup{compatibility=false}
%\usepackage{microtype}
%\usepackage[numbers]{natbib}
%\usepackage{subfig} % make it possible to include more than one captioned figure/table in a single float
% These packages are all incorporated in the memoir class to one degree or another...

\definecolor{codegreen}{rgb}{0,0.6,0}
\definecolor{codegray}{rgb}{0.5,0.5,0.5}
\definecolor{codepurple}{rgb}{0.58,0,0.82}
\definecolor{backcolour}{rgb}{0.95,0.95,0.92}
\lstdefinestyle{mystyle}{
	backgroundcolor=\color{backcolour},   commentstyle=\color{codegreen},
	keywordstyle=\color{magenta},
	numberstyle=\tiny\color{codegray},
	stringstyle=\color{codepurple},
	basicstyle=\footnotesize,
	breakatwhitespace=false,         
	breaklines=true,                 
	captionpos=b,                    
	keepspaces=true,                 
	numbers=left,                    
	numbersep=5pt,                  
	showspaces=false,                
	showstringspaces=false,
	showtabs=false,                  
	tabsize=2,
	literate=
	{á}{{\'a}}1 {é}{{\'e}}1 {í}{{\'i}}1 {ó}{{\'o}}1 {ú}{{\'u}}1
	{Á}{{\'A}}1 {É}{{\'E}}1 {Í}{{\'I}}1 {Ó}{{\'O}}1 {Ú}{{\'U}}1
	{à}{{\`a}}1 {è}{{\`e}}1 {ì}{{\`i}}1 {ò}{{\`o}}1 {ù}{{\`u}}1
	{À}{{\`A}}1 {È}{{\'E}}1 {Ì}{{\`I}}1 {Ò}{{\`O}}1 {Ù}{{\`U}}1
	{ä}{{\"a}}1 {ë}{{\"e}}1 {ï}{{\"i}}1 {ö}{{\"o}}1 {ü}{{\"u}}1
	{Ä}{{\"A}}1 {Ë}{{\"E}}1 {Ï}{{\"I}}1 {Ö}{{\"O}}1 {Ü}{{\"U}}1
	{â}{{\^a}}1 {ê}{{\^e}}1 {î}{{\^i}}1 {ô}{{\^o}}1 {û}{{\^u}}1
	{Â}{{\^A}}1 {Ê}{{\^E}}1 {Î}{{\^I}}1 {Ô}{{\^O}}1 {Û}{{\^U}}1
	{ã}{{\~a}}1 {Ã}{{\~A}}1
	{ç}{{\c c}}1 {Ç}{{\c C}}1
}

\lstset{style=mystyle}
\iffalse

\lstdefinelanguage{python}{
	keywords={typeof, null, catch, switch, in, int, str, float, self},
	%keywordstyle=\color{ForestGreen}\bfseries,
	ndkeywords={boolean, throw, import},
	ndkeywords={return, class, if ,elif, endif, while, do, else, True, False , catch, def},
	%ndkeywordstyle=\color{BrickRed}\bfseries,
	i%dentifierstyle=\color{black},
	sensitive=false,
	comment=[l]{\#},
	morecomment=[s]{/*}{*/},
	%commentstyle=\color{purple}\ttfamily,
	%stringstyle=\color{red}\ttfamily,
}
\lstset{
	basicstyle=\scriptsize\ttfamily,
	tabsize=2,
	frame=single,
	breaklines=true,
	breakatwhitespace=true,
	xleftmargin=0cm,
	xrightmargin=0cm,
	literate=
		{á}{{\'a}}1 {é}{{\'e}}1 {í}{{\'i}}1 {ó}{{\'o}}1 {ú}{{\'u}}1
		{Á}{{\'A}}1 {É}{{\'E}}1 {Í}{{\'I}}1 {Ó}{{\'O}}1 {Ú}{{\'U}}1
		{à}{{\`a}}1 {è}{{\`e}}1 {ì}{{\`i}}1 {ò}{{\`o}}1 {ù}{{\`u}}1
		{À}{{\`A}}1 {È}{{\'E}}1 {Ì}{{\`I}}1 {Ò}{{\`O}}1 {Ù}{{\`U}}1
		{ä}{{\"a}}1 {ë}{{\"e}}1 {ï}{{\"i}}1 {ö}{{\"o}}1 {ü}{{\"u}}1
		{Ä}{{\"A}}1 {Ë}{{\"E}}1 {Ï}{{\"I}}1 {Ö}{{\"O}}1 {Ü}{{\"U}}1
		{â}{{\^a}}1 {ê}{{\^e}}1 {î}{{\^i}}1 {ô}{{\^o}}1 {û}{{\^u}}1
		{Â}{{\^A}}1 {Ê}{{\^E}}1 {Î}{{\^I}}1 {Ô}{{\^O}}1 {Û}{{\^U}}1
		{ã}{{\~a}}1 {Ã}{{\~A}}1
		{ç}{{\c c}}1 {Ç}{{\c C}}1
}
\fi

% For Computer Modern:
%\def\Cpp{{C\nolinebreak[4]\hspace{-.05em}\raisebox{.4ex}{\tiny\bf ++}}}
% For Linux Libertine G
\def\Cpp{{C\nolinebreak[4]\raisebox{.20ex}{\small\bf++}}}

\newcommand{\todo}[1]{\textsf{\color{red}#1}}

\input{macros/bugcaption}

%%% END Article customizations

\title{Automatização do Gerenciamento e Implantação de Instâncias AWS EC2 para Ensino de Computação Distribuída}
\author{Augusto Contini Bernardi}{Cezar}
\course{Curso de Ciência da Computação}
\altcourse{Curso de Ciência da Computação}
\institute{Centro de Tecnologia}
\degree{Bacharel em Ciência da Computação}

%TODO número, que?
\trabalhoNumero{396}
\advisor[Prof.]{Dr.}{Lima}{João Vicente Ferreira}

\committee[Prof. Dr.]{SobrenomeBanca}{NomeBanca}{UFSM}
\committee[MSc.]{SobrenomeBanca}{NomeBanca}{UFSM}

\date{dia}{mês}{2016}

\keyword{Cloud Computing}
\keyword{Computação Distribuída}
\keyword{Orquestração}
\keyword{Ensino}

%\date{} % Activate to display a given date or no date (if empty), otherwise the current date is printed

\begin{document}
\maketitle

% ******************************************************############################################################
% ******************************************************############################################################
%% ######################################\includepdf[pages={1}]{aprove.pdf} #######################################
% ******************************************************############################################################
% ******************************************************############################################################

\chapter*{Agradecimentos}
Valeu gurizada

\begin{abstract}
Resuminho
\end{abstract}

%TODO Change Monnth
\begin{englishabstract}
	{Deployment and Management Automation of AWS EC2 Instances for Distributed Computing Tutorship}
	{Undergraduate Program in Computer Science}
	{Cloud Computing. Orchestration, Tutorship, Distributed Computing}
	{May}
	{th}
	
Abstract
	
\end{englishabstract}


\tableofcontents
\listoffigures

\setlength{\baselineskip}{1.5\baselineskip}

\chapter{Introdução}

%TODO 
Intro

\section{Objetivos}

Desenvolver um modelo de infraestrutura para Amazon Web Services focado no processo de ensino de programação paralela e distribuída, de forma que possa ser facilmente adaptado a mudanças conforme desejar o educador.

\section{Justificativa}

O ensino de computação distribuída será beneficiado significativamente, abrindo oportunidades para exploração e experimentação por parte dos alunos.

Também facilitará o trabalho do educador, o qual terá um ambiente totalmente controlado e replicável para repassar o funcionamento das ferramentas a serem ensinadas.

\section{Organização do texto}

Este trabalho está organizado da seguinte forma: O capítulo 2 apresenta fundamentação, ferramentas e trabalhos relacionados que fazem parte do tema e da proposta de solução do trabalho.

O capítulo 3 detalha a implementação do trabalho, apresentando o processo de desenvolvimento da solução e como as ferramentas apresentadas no capítulo 2 foram utilizadas.

No capítulo 4 são apresentados os resultados do trabalho: Como utilizar a ferramenta e o que o ambiente implantado disponibiliza para desenvolvimento. E por fim, no capítulo 5, apresentam-se as considerações finais e conclusões do trabalho.

\chapter{Fundamentos e Revisão de Literatura}

Para a realização deste trabalho foi feita uma pesquisa de opinião dos educadores. Tal pesquisa foi feita com o intuito de averiguar a aceitação em introduzir um novo método no processo de ensino. Além disso, foi necessário averiguar quais ferramentas, frameworks, bibliotecas e aplicativos são necessários para o andamento do conteúdo a ser ensinado.

A plataforma de IaaS da Amazon foi escolhida por ter infraestrutura física própria e prover acesso a educadores e alunos \cite{awsedu}, além de escalabilidade e ferramentas que aprimoram a replicabilidade do trabalho. Para replicabilidade, será feito o uso do Amazon CloudFormation \cite{awscf}, o qual recebe um arquivo JSON descrevendo a pilha a ser inicializada e a implanta. A pilha a ser implantada utilizará os serviços de Elastic Compute Cloud (EC2)\cite{awsec2} e Virtual Private Cloud\cite{vpc} para melhor organização e acesso.

Assim como a maioria dos serviços disponíveis para nuvem, o AWS tem diferentes valores associados aos seus recursos, incluindo uma nível de uso gratuíto com limitações \cite{ec2price}.

\section{Amazon Elastic Compute Cloud}

Amazon Elastic Compute Cloud (EC2) faz parte do modelo de IaaS da Amazon, provendo recursos de computação em nuvem. O serviço de EC2 conta com diversos tipos de instâncias que melhor se adequam a diferentes propósitos dentro da computação. Esses tipos são:

\begin{itemize}
\item Uso Geral
	\begin{itemize}
	\item T2 .. descrição? Tabela?
	\item M4
	\item M3
	\end{itemize}
\item Otimizadas para computação
	\begin{itemize}
	\item C4
	\item C3
	\end{itemize}
\item Otimizadas para memória
	\begin{itemize}
	\item R3
	\end{itemize}
\item GPU
	\begin{itemize}
	\item G2
	\end{itemize}
\item Otimizadas para armazenamento
	\begin{itemize}
	\item I2
	\item D2
	\end{itemize}
\end{itemize}

Cada tipo e sub-tipo de instância tem um valor de utilização por hora diferente, crescendo consideravelmente ao se utilizar recursos melhores.

Além dos tipos de instância, estão disponíveis, na forma de AMIs \cite{ami}, diversos tipos de imagens de máquinas virtuais, baseados em diferentes sistemas operacionais. Cada uma das AMIs tem características que as tornam melhores para determinadas tarefas computacionais, geralmente diferenciadas por recursos pré-instalados e configurados.

\section{Amazon Virtual Private Cloud}

Um recurso do AWS feito para possibilitar a divisão de outros recursos, principalmente EC2, em subnets. Com isso, é possível definir um ponto único de acesso aos demais recursos, aumentando a organização, escalabilidade e segurança da infraestrutura.

\section{Amazon CloudFormation}

Oferece um modo simples de implantar e gerenciar um grupo de recursos do AWS, de forma previsível e replicável. O uso dessa ferramenta pode ser feita de duas formas principais, sendo elas:
\begin{itemize}
\item{Declaração estruturada em JSON}
Através de um arquivo de texto estruturado é possível descrever toda a infraesturura a ser criada, com parâmetros variáveis ou estáticos para cada recurso da pilha.

\item{CloudFormation Designer}
Uma interface gráfica que disponibiliza o drag-and-drop dos recursos afim de facilitar ainda mais a criação de uma pilha de recursos AWS. 
\end{itemize}

\section{??? tchê}

E agora?

\section{Trabalhos Relacionados}

Existem outras ferramentas para facilitar a implantação de ambientes em EC2. Cada um desses projetos tem um foco diferente dos demais no que se refere a orquestração do ambiente, e normalmente requerem o uso de mais ferramentas auxiliares para o pleno funcionamento.

\begin{itemize}
	\item Stanford Elliot Slaughter \cite{stanford} - Baseado em um script Python para automatizar a implantação de diferentes tipos de clusters. Cada um desses tipos contém as ferramentas e ambiente preparado para um conteúdo específico dentre threads, STM, MapReduce (Hadoop), OpenMP e GPGPU (CUDA). A forma como o ambiente é implantada, e quais ferramentas instaladas, pode ser alterada por meio de modificações no script e nos arquivos de configuração.
	
	\item StarCluster \cite{starcluster} - Provê uma interface em CLI para criação, configuração, gerência e acesso a clusters na Amazon EC2. Este projeto automatiza a implantação em si, mas não os passos de configuração do ambiente e de ferramentas disponíveis para desenvolvimento.
	
	\item Gjovik University College - Erik Hjelmas - Bláblabla
\end{itemize}


\chapter{Desenvolvimento}

Resuminho das sessões

\section{Ambiente de rede}

A metodologia de desenvolvimento do projeto foi incremental, primeiramente focando em criar um ambiente base onde as intâncias serão posicionadas. Esse ambiente base foi feito em volta de um VPC. Primeiramente, estabeleceu-se uma rede para endereços privados sob a qual todas as instâncias estarão alocadas. Decidiu-se pelo uso da rede 192.168.0.0/24 por ser um espaço de rede comumente utilizado para o propósito de redes privadas, além de fornecer endereços suficientes para a andamento do projeto.

Esta rede foi subdividida em duas, sendo as subredes 192.168.0.0/28 para uso das instâncias de computação e 192.168.240.0/28 para a instâncias com acesso externo. A instãncia colocada sob a subrede pública além de funcionar como \emph{bastion} de acesso a rede privada com as intâncias de computação, também funciona como NAT (Networking Address Translation) para citadas instâncias. Tal medida foi tomada pois facilita o controle sobre endereçamento e organização dentro do VPC.

%TODO
/* TODO: Regiões e AMIs? */

Como boa prática de segurança, foram associados grupos de segurança as subredes, tornando disponível o acesso a redes externas somente pela intância NAT, também apenas permitindo o acesso SSH via esta e demais nodos da subrede. Além dos grupos de segurança, tabelas de roteamento foram implantadas para direcionamento do tráfego de rede aos respectivos gateways. Do ponto de vista da subrede pública, esse gateway é um recurso especial da Amazon que permite o acesso a internet, chamado Internet Gateway, e do ponto de vista da subrede privada o gateway é a própria instância NAT.

%TODO
/* TODO: Figuras do código?
*Imagem representativa do ambiente? */

\section{Instâncias}

Para realização do projeto foram utilizadas instâncias do tipo \emph{t2.micro}, as quais se encaixam no plano de uso livre da AWS. Esse tipo pode ser alterado no momento da implantação do modelo, com a ressalva de que os servidores AWS da América do Sul não suportam o tipo \emph{g2}, com GPUs. Porém, essas instâncias podem ser utilizadas se a região escolhida for outra, como \emph{us-east}.

Para configuração de arquivos e instalação de pacotes necessários foram utilizados as ferramentas disponibilizadas pelo recurso Init do CloudFormation. Dentre as opções disponíveis, foram utilizadas as seguites:

\begin{itemize}
\item Packages - Provê suporte aos gerênciadores de instalação apt, msi, python, rpm, rubygems e yum. 
\item Files - Suporte para criação de arquivos e diretóríos, assim como seus conteúdos.
\item Commands - Disponibiliza a execução de comandos \emph{bash}, no caso de instâncias Linux.
\end{itemize}

Essas opção são sempre executadas pelo CloudFormation na ordem descrita, evitando a edição de pacotes ou arquivos ainda inexistentes.

\subsection{Acesso SSH}

O primeiro desafio encontrado nesta etapa foi a simplificação da configuração do acesso SSH entre as instãncias, necessário a execução de algoritmos distribuídos com MPI. Por padrão, instâncias EC2 já têm a chave pública referente ao \emph{key pair} configurado previamente a aplicação da pilha, porém, a chave privada está apenas disponível em arquivo PPK e é utilizada da máquina do usuário para acesso as instância pública. Isto não permite que as instâncias tenham acesso entre si via SSH, então foi necessário requerer a chave privada como \emph{string} no momento de implantação, salvá-las nas VMs e reorganizar a estrutura do arquivo com um script python, pois a quebras de linha são perdidas e são necessárias para a correta interpretação da chave.

\begin{lstlisting}[language=Python]
import re
input = open("/home/ec2-user/.ssh/id_rsa.in").read()
bgnEnd = re.findall('-----[A-Z\s]+-----',input)
content = re.findall('-\s(.+?)\s-',input)
lines = content[0].split(" ")
outputFile = open("/home/ec2-user/.ssh/id_rsa", "w")
outputFile.write(bgnEnd[0] + '\n')
[outputFile.write(line + '\n') for line in lines]
outputFile.write(bgnEnd[1])
outputFile.close()
\end{lstlisting}

Outra modificação aplicada foi a edição do arquivo \verb|/etc/hosts| para adicionar \emph{alias} aos endereços das instâncias da subrede.

\begin{lstlisting}
"AWS::CloudFormation::Init" : {
	"config" : {
		"files" : {
			"/etc/hosts" : {
				"content" : {
					"Fn::Join" : ["\n", [
					"127.0.0.1   localhost   localhost",
					{"Fn::FindInMap":["IpAddressConfig", "Node01", "IP" ]},"  node01  node01",
					{"Fn::FindInMap":["IpAddressConfig", "Node02", "IP" ]},"  node02  node02",
					{"Fn::FindInMap":["IpAddressConfig", "Node03", "IP" ]},"  node03  node03",
					{"Fn::FindInMap":["IpAddressConfig", "Node04", "IP" ]},"  node04  node04",
					{"Fn::FindInMap":["IpAddressConfig", "Node05", "IP" ]},"  node05  node05"]]
				}
			}
		}
	}
}
\end{lstlisting}

\subsection{Instalação de Pacotes}

%TODO outros pacotes
A imagem utilizada nas VMs já possui repositórios yum configurados, diponibilizando a instalação dos pacotes openmpi, gcc e g++ /*até agora*/.

\subsubsection{OpenMPI}

A configuração desse pacote começa com a exportação de variáveis de ambiente, indicando os \emph{wrappers} de compilação e execução. Para isso foi utilizado o recurso de execução de comandos dos Init.

\begin{lstlisting}
"AWS::CloudFormation::Init" : {
	"config" : {
		"commands" : {
			"exportOpenMPIBin" : {
				"command" : "echo \"export PATH=/usr/lib64/openmpi/bin:$PATH\" >> /home/ec2-user/.bashrc"
			},
			"exportOpenMPILib" : {
				"command" : "echo \"export LD_LIBRARY_PATH=/usr/lib64/openmpi/lib:$LD_LIBRARY_PATH\" >> /home/ec2-user/.bashrc"
			}
		}
	}
}
\end{lstlisting}



\chapter{Resultados}

Funciona assim ó:

\chapter{Conclusão}

É isso champs.


\setlength{\baselineskip}{\baselineskip}
\bibliographystyle{abnt}
\bibliography{references}

\end{document}
