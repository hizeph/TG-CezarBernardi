% !TEX TS-program = pdfLaTeX+MakeIndex+BibTeX
% !TEX encoding = UTF-8 Unicode

\PassOptionsToPackage{unicode}{hyperref}
\PassOptionsToPackage{naturalnames}{hyperref}

\documentclass[tg]{mdtufsm}

\usepackage[T1]{fontenc}
\usepackage{fix-cm}
\usepackage{url}
\usepackage{times, color}
\usepackage[utf8]{inputenc}
\usepackage{graphicx}
\usepackage{amsmath,latexsym,amssymb}
%\usepackage[hidelinks]{hyperref}
\usepackage[hidelinks,
            bookmarksopen=true,linktoc=none,colorlinks=true,
            linkcolor=black,citecolor=black,filecolor=magenta,urlcolor=blue,
            pdftitle={Automatização do Gerenciamento e Implantação de Instâncias AWS EC2 para Ensino de Computação Distribuída},
            pdfauthor={Cezar Augusto Contini Bernardi},
            pdfsubject={Trabalho de Graduação},
            pdfkeywords={Cloud computing, orquestração, computação distribuída, ensino, UFSM}
            ]{hyperref}
            
%\usepackage[brazilian]{babel}

%\usepackage{fontspec}
%\setmainfont{Linux Libertine G}

%%% PAGE DIMENSIONS
\usepackage[inner=30mm,outer=20mm,top=30mm,bottom=20mm]{geometry} 
\usepackage{epstopdf}
\usepackage{graphicx}
\usepackage{pdfpages}
% \geometry{margin=2in} % for example, change the margins to 2 inches all round
% \geometry{landscape} % set up the page for landscape

% \usepackage[parfill]{parskip} % Activate to begin paragraphs with an empty line rather than an indent

%%% PACKAGES
%\usepackage{amsfonts}
\usepackage{color}
%\usepackage{booktabs} % for much better looking tables
%\usepackage{array} % for better arrays (eg matrices) in maths
%\usepackage{paralist} % very flexible & customisable lists (eg. enumerate/itemize, etc.)
\usepackage{verbatim} % adds environment for commenting out blocks of text & for better verbatim
\usepackage{listings}
\usepackage{parcolumns}
\usepackage{siunitx}
%\usepackage{verbatim} % adds environment for commenting out blocks of text & for better verbatim
\usepackage{subcaption}
\captionsetup{compatibility=false}
%\usepackage{microtype}
%\usepackage[numbers]{natbib}
%\usepackage{subfig} % make it possible to include more than one captioned figure/table in a single float
% These packages are all incorporated in the memoir class to one degree or another...

\definecolor{codegreen}{rgb}{0,0.6,0}
\definecolor{codegray}{rgb}{0.5,0.5,0.5}
\definecolor{codepurple}{rgb}{0.58,0,0.82}
\definecolor{backcolour}{rgb}{0.95,0.95,0.92}
\lstdefinestyle{mystyle}{
	backgroundcolor=\color{backcolour},   commentstyle=\color{codegreen},
	keywordstyle=\color{magenta},
	numberstyle=\tiny\color{codegray},
	stringstyle=\color{codepurple},
	basicstyle=\footnotesize,
	breakatwhitespace=false,         
	breaklines=true,                 
	captionpos=b,                    
	keepspaces=true,                 
	numbers=left,                    
	numbersep=5pt,                  
	showspaces=false,                
	showstringspaces=false,
	showtabs=false,                  
	tabsize=2,
	literate=
	{á}{{\'a}}1 {é}{{\'e}}1 {í}{{\'i}}1 {ó}{{\'o}}1 {ú}{{\'u}}1
	{Á}{{\'A}}1 {É}{{\'E}}1 {Í}{{\'I}}1 {Ó}{{\'O}}1 {Ú}{{\'U}}1
	{à}{{\`a}}1 {è}{{\`e}}1 {ì}{{\`i}}1 {ò}{{\`o}}1 {ù}{{\`u}}1
	{À}{{\`A}}1 {È}{{\'E}}1 {Ì}{{\`I}}1 {Ò}{{\`O}}1 {Ù}{{\`U}}1
	{ä}{{\"a}}1 {ë}{{\"e}}1 {ï}{{\"i}}1 {ö}{{\"o}}1 {ü}{{\"u}}1
	{Ä}{{\"A}}1 {Ë}{{\"E}}1 {Ï}{{\"I}}1 {Ö}{{\"O}}1 {Ü}{{\"U}}1
	{â}{{\^a}}1 {ê}{{\^e}}1 {î}{{\^i}}1 {ô}{{\^o}}1 {û}{{\^u}}1
	{Â}{{\^A}}1 {Ê}{{\^E}}1 {Î}{{\^I}}1 {Ô}{{\^O}}1 {Û}{{\^U}}1
	{ã}{{\~a}}1 {Ã}{{\~A}}1
	{ç}{{\c c}}1 {Ç}{{\c C}}1
}

\lstset{style=mystyle}
\iffalse

\lstdefinelanguage{python}{
	keywords={typeof, null, catch, switch, in, int, str, float, self},
	%keywordstyle=\color{ForestGreen}\bfseries,
	ndkeywords={boolean, throw, import},
	ndkeywords={return, class, if ,elif, endif, while, do, else, True, False , catch, def},
	%ndkeywordstyle=\color{BrickRed}\bfseries,
	i%dentifierstyle=\color{black},
	sensitive=false,
	comment=[l]{\#},
	morecomment=[s]{/*}{*/},
	%commentstyle=\color{purple}\ttfamily,
	%stringstyle=\color{red}\ttfamily,
}
\lstset{
	basicstyle=\scriptsize\ttfamily,
	tabsize=2,
	frame=single,
	breaklines=true,
	breakatwhitespace=true,
	xleftmargin=0cm,
	xrightmargin=0cm,
	literate=
		{á}{{\'a}}1 {é}{{\'e}}1 {í}{{\'i}}1 {ó}{{\'o}}1 {ú}{{\'u}}1
		{Á}{{\'A}}1 {É}{{\'E}}1 {Í}{{\'I}}1 {Ó}{{\'O}}1 {Ú}{{\'U}}1
		{à}{{\`a}}1 {è}{{\`e}}1 {ì}{{\`i}}1 {ò}{{\`o}}1 {ù}{{\`u}}1
		{À}{{\`A}}1 {È}{{\'E}}1 {Ì}{{\`I}}1 {Ò}{{\`O}}1 {Ù}{{\`U}}1
		{ä}{{\"a}}1 {ë}{{\"e}}1 {ï}{{\"i}}1 {ö}{{\"o}}1 {ü}{{\"u}}1
		{Ä}{{\"A}}1 {Ë}{{\"E}}1 {Ï}{{\"I}}1 {Ö}{{\"O}}1 {Ü}{{\"U}}1
		{â}{{\^a}}1 {ê}{{\^e}}1 {î}{{\^i}}1 {ô}{{\^o}}1 {û}{{\^u}}1
		{Â}{{\^A}}1 {Ê}{{\^E}}1 {Î}{{\^I}}1 {Ô}{{\^O}}1 {Û}{{\^U}}1
		{ã}{{\~a}}1 {Ã}{{\~A}}1
		{ç}{{\c c}}1 {Ç}{{\c C}}1
}
\fi

% For Computer Modern:
%\def\Cpp{{C\nolinebreak[4]\hspace{-.05em}\raisebox{.4ex}{\tiny\bf ++}}}
% For Linux Libertine G
\def\Cpp{{C\nolinebreak[4]\raisebox{.20ex}{\small\bf++}}}

\newcommand{\todo}[1]{\textsf{\color{red}#1}}

\input{macros/bugcaption}

%%% END Article customizations

\title{Automatização do Gerenciamento e Implantação de Instâncias AWS EC2 para Ensino de Computação Distribuída}
\author{Augusto Contini Bernardi}{Cezar}
\course{Curso de Ciência da Computação}
\altcourse{Curso de Ciência da Computação}
\institute{Centro de Tecnologia}
\degree{Bacharel em Ciência da Computação}

%TODO número, que?
\trabalhoNumero{396}
\advisor[Prof.]{Dr.}{Lima}{João Vicente Ferreira}

\committee[Prof. Dr.]{SobrenomeBanca}{NomeBanca}{UFSM}
\committee[MSc.]{SobrenomeBanca}{NomeBanca}{UFSM}

\date{dia}{mês}{2016}

\keyword{Cloud Computing}
\keyword{Computação Distribuída}
\keyword{Orquestração}
\keyword{Ensino}

%\date{} % Activate to display a given date or no date (if empty), otherwise the current date is printed

\begin{document}
\maketitle

% ******************************************************############################################################
% ******************************************************############################################################
%% ######################################\includepdf[pages={1}]{aprove.pdf} #######################################
% ******************************************************############################################################
% ******************************************************############################################################

\chapter*{Agradecimentos}
Valeu gurizada

\begin{abstract}
Resuminho
\end{abstract}

%TODO Change Monnth
\begin{englishabstract}
	{Deployment and Management Automation of AWS EC2 Instances for Distributed Computing Tutorship}
	{Undergraduate Program in Computer Science}
	{Cloud Computing. Orchestration, Tutorship, Distributed Computing}
	{May}
	{th}
	
Abstract
	
\end{englishabstract}


\tableofcontents
\listoffigures

\setlength{\baselineskip}{1.5\baselineskip}

\chapter{Introdução}

%TODO 
Intro

\section{Objetivos}

Desenvolver um modelo de infraestrutura para Amazon Web Services focado no processo de ensino de programação paralela e distribuída, de forma que possa ser facilmente adaptado a mudanças conforme desejar o educador.

\section{Justificativa}

O ensino de computação distribuída será beneficiado significativamente, abrindo oportunidades para exploração e experimentação por parte dos alunos.

Também facilitará o trabalho do educador, o qual terá um ambiente totalmente controlado e replicável para repassar o funcionamento das ferramentas a serem ensinadas.

\section{Organização do texto}
%TODO depois, bem depois
Algo assim:

Este trabalho está organizado da seguinte forma: O capítulo 2 apresenta fundamentação, ferramentas e trabalhos relacionados que fazem parte do tema e da proposta de solução do trabalho.

O capítulo 3 detalha a parte lógica e a implementação do trabalho, apresentando modelos de dados, usuários, e fluxo de execução, o processo de desenvolvimento da solução e como as ferramentas apresentadas no capítulo 2 foram utilizadas.

No capítulo 4 são apresentados os resultados do trabalho: Em que ambiente ele foi disponibilizado, como foi testado e como funciona seu fluxo de trabalho. Isso é demonstrado através de telas que ilustram a utilização do sistema por um usuário, o preenchimento de um formulário de cadastramento, até o \emph{download} de um resultado gerado por uma execução requisitada por ele. E por fim, no capítulo 5, apresentam-se as considerações finais e conclusões do trabalho.

\chapter{Fundamentos e Revisão de Literatura}

Para a realização deste trabalho foi feita uma pesquisa de opinião dos educadores. Tal pesquisa foi feita com o intuito de averiguar a aceitação em introduzir um novo método no processo de ensino. Além disso, foi necessário averiguar quais ferramentas, frameworks, bibliotecas e aplicativos são necessários para o andamento do conteúdo a ser ensinado.

A plataforma de IaaS da Amazon foi escolhida por ter infraestrutura física própria e prover acesso a educadores e alunos \cite{awsedu}, além de escalabilidade e ferramentas que aprimoram a replicabilidade do trabalho. Para replicabilidade, será feito o uso do Amazon CloudFormation \cite{awscf}, o qual recebe um arquivo JSON descrevendo a pilha a ser inicializada e a implanta. A pilha a ser implantada utilizará os serviços de Elastic Compute Cloud (EC2)\cite{awsec2} e Virtual Private Cloud\cite{vpc} para melhor organização e acesso.

Assim como a maioria dos serviços disponíveis para nuvem, o AWS tem diferentes valores associados aos seus recursos, incluindo uma nível de uso gratuíto com limitações \cite{ec2price}.

\section{Amazon Elastic Compute Cloud}

Amazon Elastic Compute Cloud (EC2) faz parte do modelo de IaaS da Amazon, provendo recursos de computação em nuvem. O serviço de EC2 conta com diversos tipos de instâncias que melhor se adequam a diferentes propósitos dentro da computação. Esses tipos são:

\begin{itemize}
\item Uso Geral
	\begin{itemize}
	\item T2 .. descrição? Tabela?
	\item M4
	\item M3
	\end{itemize}
\item Otimizadas para computação
	\begin{itemize}
	\item C4
	\item C3
	\end{itemize}
\item Otimizadas para memória
	\begin{itemize}
	\item R3
	\end{itemize}
\item GPU
	\begin{itemize}
	\item G2
	\end{itemize}
\item Otimizadas para armazenamento
	\begin{itemize}
	\item I2
	\item D2
	\end{itemize}
\end{itemize}

Cada tipo e sub-tipo de instância tem um valor de utilização por hora diferente, crescendo consideravelmente ao se utilizar recursos melhores.

Além dos tipos de instância, estão disponíveis, na forma de AMIs \cite{ami}, diversos tipos de imagens de máquinas virtuais, baseados em diferentes sistemas operacionais. Cada uma das AMIs tem características que as tornam melhores para determinadas tarefas computacionais, geralmente diferenciadas por recursos pré-instalados e configurados.

\section{Amazon Virtual Private Cloud}

Um recurso do AWS feito para possibilitar a divisão de outros recursos, principalmente EC2, em subnets. Com isso, é possível definir um ponto único de acesso aos demais recursos, aumentando a organização, escalabilidade e segurança da infraestrutura.

\section{Amazon CloudFormation}

Oferece um modo simples de implantar e gerenciar um grupo de recursos do AWS, de forma previsível e replicável. O uso dessa ferramenta pode ser feita de duas formas principais, sendo elas:
\begin{itemize}
\item{Declaração estruturada em JSON}
Através de um arquivo de texto estruturado é possível descrever toda a infraesturura a ser criada, com parâmetros variáveis ou estáticos para cada recurso da pilha.

\item{CloudFormation Designer}
Uma interface gráfica que disponibiliza o drag-and-drop dos recursos afim de facilitar ainda mais a criação de uma pilha de recursos AWS. 
\end{itemize}

\section{??? tchê}

E agora?

\section{Trabalhos Relacionados}

Relacionados

\begin{itemize}
	\item Stanford Elliot Slaughter \cite{stanford} - Bláblabla
	
	\item Gjovik University College - Erik Hjelmas - Bláblabla
\end{itemize}

StarCluster?
\iffalse
Existem também portais que a interação que o usuário faz com o sistema é a produção de código fonte em determinada linguagem. O objetivo desses portais varia entre o ensino de novas técnicas de programação, ensino das linguagens e desafios de desempenho de algoritmos. Para as situações de ensino, o sistema pode instruir o usuário como resolver determinado problema e solicitar que o mesmo resolva algum problema semelhante, provando que aprendeu o conteúdo proposto.
Para desafios de algoritmos, o usuário envia a sua solução proposta para o sistema executar, e após a execução, recebe uma avaliação que é publicada em um \emph{rank}, promovendo competição entre os usuários. 
Exemplos de portais que seguem essa metodologia:
\begin{itemize}
	\item HackerRank \cite{hackerrank} - Este site desafia os usuários a encontrem as melhores soluções para desafios computacionais que podem ser resolvidos em várias linguagens de programação. Cada solução proposta por um usuário é avaliada de acordo com o resultado do algoritmo enviado e seu tempo de execução, essa avaliação é publicada em um \emph{rank} contendo as demais avaliações da comunidade, criando uma competição associada ao desafio proposto.
	\item Codecademy \cite{codecademy} - Diferente do exemplo anterior, a proposta do Codecademy é o ensino de linguagens e paradigmas de programação para seus usuários. Através da plataforma, o usuário pode selecionar, a partir de uma lista de linguagens, qual linguagem de programação deseja aprender, e seguir um tutorial contendo exemplos e desafios de programação da opção escolhida.
	Os códigos gerados são testados na plataforma e se os resultados forem conforme os esperados de cada exercício, é liberado um novo tópico de aprendizagem até que o curso seja finalizado.
	\item CodeinGame \cite{codeingame} - Essa plataforma pode ser avaliada como um intermediário entre os outros dois sites citados acima. Assim como o Codecademy, o propósito do CodeinGame é o ensino de técnicas e linguagens de programação, porém, ele tem possui as características competitivas do HackerRank pois também oferece placares de liderança e "batalhas" de códigos (principalmente ligados a área de inteligência artificial), aonde dois participantes cadastram seus algoritmos que serão avaliados para resolução de um mesmo problema.
\end{itemize}

Outro modelo de portal pode ser observado no Algorithmia \cite{algor}, uma plataforma que serve de intermediário entre desenvolvedores e clientes do mercado de algoritmos. O desenvolvedor pode disponibilizar um algoritmo no portal, e cobrar royalties por sua utilização, enquanto o portal faz a execução do algoritmo solicitado pelo cliente, e cobra uma taxa pelo serviço por execução.
\fi

\chapter{Desenvolvimento}

BAM!


\chapter{Conclusão}

É isso champs.


\setlength{\baselineskip}{\baselineskip}
\bibliographystyle{abnt}
\bibliography{references}

\end{document}
